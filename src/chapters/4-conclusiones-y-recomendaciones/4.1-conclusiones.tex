\section{Conclusiones}

El presente trabajo de titulación se desarrolló con el objetivo de proponer una solución tecnológica que permita mejorar la gestión de la administración del Conjunto habitacional  {\textquotedblleft}El Portal de la Viña{\textquotedblright}, a través de la implementación de un sistema web y móvil que permita a la directiva y a los residentes gestionar de una manera más ágil y cómoda los procesos administrativos, así como proporcionar un acceso más rápido a la información del conjunto habitacional.

\begin{itemize}
    \item Con la ayuda de las entrevistas y encuestas se pudo identificar los problemas principales que enfrentaba la directiva del conjunto habitacional, con lo cual se logró proponer una solución tecnológica que permita mejorar la gestión y el acceso a la información.
    La entrevista con el presidente de la directiva fue en mayor parte la que permitió identificar los problemas administrativos, tales como la deficiencia en la toma de asistencia en las asambleas, la dificultad de acceder a la información de las obligaciones financieras de los residentes y la mala organización de la información de los parqueaderos.
    Con estos problemas analizados se pudo evidenciar la necesidad de tener un sistema que permita gestionar de manera eficiente la información del conjunto habitacional.
    \item El uso de frameworks de desarrollo ha demostrado que son herramientas que permiten acelerar el proceso de desarrollo de software, ya que proveen de un marco de trabajo ya definido con el cual se puede agilizar de forma considerable el desarrollo de cualquier software, además de que existe una gran cantidad de documentación y una comunidad activa que puede ayudar a resolver problemas que se presenten durante el desarrollo.
    Sin embargo, la elección del framework adecuado dependerá de las necesidades del proyecto y de la experiencia de las personas involucradas en el desarrollo.
    \item La metodología de desarrollo de software debe ser seleccionada de acuerdo a la naturaleza del proyecto, ya que depende directamente de los requerimientos y del equipo de trabajo.
    En el caso partícular de este proyecto se pudo observar que el uso de una metodología ágil como RAD permitió tener un desarrollo rápido y con la participación activa de los usuarios, lo cual permitió tener un producto final que cumplió con las necesidades y expectativas del cliente.
    \item La implementación de las validaciones mediante el uso de la geolocalización junto con el cálculo de la distancia de Haversiné demostraron ser unas herramientas útiles para verificar la ubicación de los residentes durante las asambleas.
    Esta estrategia ha tenido un impacto positivo en la confiabilidad del proceso de asistencias y votaciones, previniendo el mal uso de la aplicación y garantizando la autenticidad de los datos durante las sesiones de asamblea.
    \item El uso de encuestas TAM permitieron evaluar la percepción de los usuarios sobre la aplicación móvil y web, lo cual permitió identificar los aspectos positivos y negativos de la aplicación.
\end{itemize}