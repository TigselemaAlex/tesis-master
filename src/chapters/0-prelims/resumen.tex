\chapter*{resumen ejecutivo}

El conjunto habitacional {\textquotedblleft}El Portal de la Viña{\textquotedblright} consta de 303 residencias, lo cual dificulta el control de los procesos administrativos por parte de la directiva, dando lugar a desafíos en la gestión de los parqueaderos, convocatorias y en tesorería, ya que estos procesos se realizan de manera manual.
\bigbreak
El presente proyecto implementa un sistema de gestión web junto con una aplicación móvil para sistematizar los procesos administrativos y financieros del conjunto habitacional. Además, se utiliza la geolocalización como mecanismo de verificación de la asistencia y las votaciones en las asambleas. Para ello, se emplea la fórmula de Haversine para calcular con precisión la distancia entre la ubicación del residente y el punto de la asamblea, asegurando así que solo los asistentes presentes puedan participar en las votaciones.
\bigbreak
El sistema de gestión del conjunto habitacional incluye módulos para la administración de propietarios, residencias, pagos, convocatorias, guardianía y parqueaderos. Se emplearon tecnologías modernas como los frameworks Angular e Ionic para el cliente web y móvil, respectivamente. Para el desarrollo de la interfaz de programación de aplicaciones (API), se utilizó el framework Spring Boot junto con PostgreSQL como gestor de base de datos.
\bigbreak
La validación del sistema implementado se realizó aplicando el modelo de aceptación de tecnología (TAM) a los residentes del conjunto habitacional, teniendo como resultado una aceptación positiva del sistema propuesto. Se espera que la implementación de este sistema mejore la eficiencia y la transparencia en la administración del conjunto habitacional.

\vfill
\textbf{Palabras clave:} Sistematización, Conjunto habitacional, Framework, Geolocalización, Angular, Ionic, Spring boot.
