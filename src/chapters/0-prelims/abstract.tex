\newpage
\chapter*{abstract}
\addcontentsline{toc}{chapter}{\bfseries \uppercase{ABSTRACT}}

The {\textquotedblleft}El Portal de la Viña{\textquotedblright} housing complex consists of 303 residences, which makes it difficult for the board of directors to control the administrative processes, giving rise to challenges in the management of parking spaces, calls, and in treasury, since these processes are carried out manually.
\bigbreak
This project implements a web management system along with a mobile application to systematize the administrative and financial processes of the housing complex. In addition, geolocation is used as a mechanism to verify attendance and voting at assemblies. To do this, the Haversine formula is used to accurately calculate the distance between the resident's location and the assembly point, thus ensuring that only those present can participate in the votes.
\bigbreak
The housing complex management system includes modules for the administration of owners, residences, payments, calls, guardianship, and parking. Modern technologies such as Angular and Ionic frameworks were used for the web and mobile client, respectively. For the development of the application programming interface (API), the Spring Boot framework was used along with PostgreSQL as a database manager.
\bigbreak
The validation of the implemented system was carried out by applying the technology acceptance model (TAM) to the residents of the housing complex, resulting in a positive acceptance of the proposed system. The implementation of this system is expected to improve efficiency and transparency in the administration of the housing complex.

\vfill
\textbf{Keywords:} Systematization, housing complex, Framework, geolocation, Angular, Ionic, Spring boot.
