\subsection{Planteamiento del problema}\label{subsec:planteamiento-del-problema}
En\cite{MoranSalazar} el autor menciona que los avances tecnológicos han sido muy relevantes en la sociedad actual, debido a que han permitido la optimización de procesos y la automatización de tareas, lo cual ha facilitado a las empresas y organizaciones ser más eficientes y productivas.
En el caso administrativo, las herramientas tecnológicas ofrecen soluciones automatizadas mediante sistemas computacionales que reemplazan a los sistemas manuales tradicionales.
\bigbreak
En\cite{INEC}, menciona que Ambato según el censo del 2022 realizado por el Instituto Nacional de Estadística y Censos(INEC) se ha registrado un total de 153948 viviendas particulares y 234 viviendas colectivas con un promedio de 3.18 habitantes por vivienda.
En estos casos, según\cite{Propiedad_Horizontal} en su Artículo 11 de la Ley de Propiedad horizontal los copropietarios tienen derecho a establecer sus propias normativas y reglamentos internos con la finalidad de poder crear modelos de gestión internos.
En este sentido la gestión de los condóminios o departamentos se realizan mediante sistemas manuales lo cual al ser conjuntos residenciales son muy ineficientes debido a la gran cantidad de viviendas que suelen tener.
Por consiguiente se hace necesario la implementación de herramientas tecnológicas que permitan agilizar y automatizar los procesos administrativos y financieros de los condominios o departamentos
\bigbreak
En\cite{FajardoFlores}, menciona que el conjunto Habitacional {\textquotedblleft}El Portal de la Viña{\textquotedblright} pese a que cuenta con un reglamento de gestión, el cual se rige por la ley de propiedad horizontal, y con un modelo de gestión administrativo, carece de una herramienta tecnológica que automatice estos procesos.
En este sentido la administración de la gerencia se realiza en gran medida mediante sistemas de hojas de cálculo en línea y físicos, lo cual dificulta la gestión de los recursos del conjunto habitacional.
En el caso de las asambleas donde se aprueban o rechazan normativas mediante votaciones es necesario tener un control de los asistentes el cual es realizado actualmente de manera escrita.
\bigbreak
De lo anteriormente expuesto la siguiente investigación se realizará en el conjunto habitacional {\textquotedblleft}El Portal de la Viña{\textquotedblright} ubicado en la ciudad de Ambato, provincia de Tungurahua, Ecuador en el periodo académico marzo 2024 - agosto 2024.