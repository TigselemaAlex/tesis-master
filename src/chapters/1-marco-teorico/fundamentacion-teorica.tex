\section{Fundamentación teórica}\label{subsec:fundamentacion-teorica}

\subsection{Condominio}\label{subsec:condominio}
Es un conjunto de viviendas que comparten áreas comunes y servicios. En\cite{montalvoDesarrolloSistemaSoftware2023, ortegaAnalisisDisenoImplementacion2017, pintoSistemaGestionAdministrativo2017,moreiraDESARROLLOSISTEMAWEB2019}, los autores mencionan que un condominio o copropiedad es un lugar, ya sea un conjunto de viviendas o terrenos, donde conviven propiedades compartidas por todos los dueños y propiedades exclusivas de cada propietario. En\cite{bravoSISTEMAINFORMACIONWEB2015}, los autores mencionan que un condominio es un desarrollo inmobiliario caracterizado por estar conformado por varios edificios o unidades de vivienda.

\subsection{Gestión administrativa}\label{subsec:gestion-administrativa}
Es el proceso de planificación, organización, dirección y control de los recursos de una organización con el fin de alcanzar sus objetivos. En\cite{montalvoDesarrolloSistemaSoftware2023, portugalSistemaInformacionPara2022, moscayzaSistemaWebPara}, los autores mencionan que es necesario desarrollar una práctica de gestión democrática y participativa para la toma de decisiones en las asambleas de condominios.

\subsection{Modelo de gestión}\label{subsec:modelo-de-gestion}
Es un conjunto de procesos que permiten la gestión de una organización. En\cite{montalvoDesarrolloSistemaSoftware2023, moscayzaSistemaWebPara, nietoPrototipoAplicacionWeb2018}, los autores mencionan que un modelo de gestión permite imitar o reproducir un arquetipo de gestión que ha sido exitoso en otras organizaciones.

\subsection{Ley de propiedad horizontal}\label{subsec:ley-de-propiedad-horizontal}
Es una ley que regula la convivencia de los propietarios de un condominio. En\cite{Propiedad_Horizontal},su artículo 11 menciona {\textquotedblleft}El Reglamento General de esta Ley establecerá un capítulo especial para precisar
los derechos y obligaciones recíprocos de los copropietarios. Los propietarios de los
diversos pisos, departamentos o locales, podrán constituir una sociedad que tenga a su
cargo la administración de los mismos. Si no lo hicieren, deberán dictar un reglamento
interno acorde con el Reglamento General{\textquotedblright}. Su artículo 12 menciona {\textquotedblleft}El Reglamento Interno de Copropiedad contendrá las normas sobre administración
y conservación de los bienes comunes, funciones que correspondan a la Asamblea de los
Copropietarios, facultades y obligaciones y forma de elección del administrador,
distribución de las cuotas de administración entre los copropietarios y todo lo que converge
a los intereses de los copropietarios y al mantenimiento y conservación del edificio.{\textquotedblright}

\subsection{Sistema web}\label{subsec:sistema-web}
Es un sistema de información que se encuentra alojado en un servidor web y que puede ser accedido mediante un navegador. En\cite{moreiraDESARROLLOSISTEMAWEB2019, ortegaAnalisisDisenoImplementacion2017, lopezPROTOTIPOSOFTWAREWEB2020, leonardoMejoraControlAsistencia2019}, los autores indican que la utilización de un sistema web permite gestionar más rápido la información y procesos de una organización residencial.
En\cite{lopezPROTOTIPOSOFTWAREWEB2020, bravoSISTEMAINFORMACIONWEB2015, montalvoDesarrolloSistemaSoftware2023, castroReconocimientoFacialGeolocalizacion2023}, los autores mencionan que los sistemas web poseen la característica de poder manipular información desde cualquier parte del mundo, siempre y cuando se tenga acceso a internet.

\paragraph{HTML5}
Es un lenguaje de marcado de hipertexto, el cual define los contenidos que formarán parte de la página web. En\cite{izaDesarrolloERPPara2023}, los autores mencionan que este lenguaje es útil para describir sintácticamente el contenido de una página web.

\paragraph{CSS3}
Es un lenguaje de hojas de estilo en cascada, el cual permite definir la presentación de los documentos HTML. En\cite{izaDesarrolloERPPara2023}, los autores mencionan que es fundamental estilar el diseño acorde al prototipo diseñado por lo cual es necesario el uso de este lenguaje.

\subsection{Sistema móvil}\label{subsec:sistema-movil}
Es un sistema de información que se encuentra alojado en un dispositivo móvil. En\cite{ortegaAnalisisDisenoImplementacion2017, leonardoMejoraControlAsistencia2019, izaDesarrolloERPPara2023}, los autores mencionan que un sistema móvil tiene características similares a un sistema web con la ventaja de la portabilidad que ofrecen los dispositivos móviles.
\bigbreak
\textbf {Android} \\
Es un sistema operativo de código abierto para dispositivos móviles, el cual es desarrollado por Google. En\cite{ortegaAnalisisDisenoImplementacion2017}, los autores señalan que según estadísticas, este sistema operativo es el más utilizado a nivel nacional en un 82.2\%.

\subsection{API}\label{subsec:api}
Es una interfaz de programación de aplicaciones, la cual permite la comunicación entre dos aplicaciones de software. En\cite{leonardoMejoraControlAsistencia2019, montalvoDesarrolloSistemaSoftware2023}, los autores menciona que la implementación de una API permite la comunicación entre el sistema web y el sistema móvil.
\bigbreak
\textbf{Google Maps API} \\
Es una API de Google que permite la integración de mapas en aplicaciones web y móviles. En\cite{izaDesarrolloERPPara2023, castroReconocimientoFacialGeolocalizacion2023}, los autores mencionan que esta API permite localizar geográficamente a los usuarios de una aplicación web o móvil.


\subsection {Gestor de bases de datos}\label{subsec:gestor-de-bases-de-datos}
Un gestor de bases de datos es una aplicación o software diseñado para crear, administrar, manipular y gestionar datos en una base de datos.\\
Mysql es un sistema de gestión de bases de datos relacional mantenido por Oracle. En\cite{moscayzaSistemaWebPara}, los autores mencionan que este gestor de base de datos trabaja perfectamente bien con sistemas web y es fácil de implementar. En\cite{izaDesarrolloERPPara2023}, los autores indican que este gestor de base de datos posee una herramienta visual fácil de utilizar en la cual se pueden crear tablas, relaciones y consultas.\\
SQL Server es un sistema de gestión de bases de datos relacional desarrollado por Microsoft. En\cite{ortegaAnalisisDisenoImplementacion2017}, los autores mencionan que este motor de bases de datos fue seleccionado debido a que es apto para dar soluciones de comercio electrónico y a la facilidad de implementación.
PostgreSQL es un sistema de gestión de bases de datos relacional de código abierto. En\cite{montalvoDesarrolloSistemaSoftware2023}, los autores indican que este gestor es compatible con el estándar SQL y posee las características de tener control de concurrencia e integridad transaccional.

\subsection{Lenguajes de programación}\label{subsec:lenguajes-de-programacion}
Un lenguaje de programación es un lenguaje formal que permite a un programador especificar de manera precisa sobre un conjunto de instrucciones para que una computadora pueda producir un resultado.
\bigbreak
\textbf {PHP} \\
Es un lenguaje de programación de código abierto, el cual es utilizado para el desarrollo de aplicaciones web. En\cite{moscayzaSistemaWebPara, izaDesarrolloERPPara2023}, el autor señala que este lenguaje tiene la facilidad de poder incrustar código PHP en código HTML, lo cual permite que el desarrollo de aplicaciones web sea más rápido y sencillo. En\cite{izaDesarrolloERPPara2023, ortegaAnalisisDisenoImplementacion2017}, los autores mencionan que este lenguaje es muy utilizado en el desarrollo de software, dado que es muy robusto.
\bigbreak
\textbf{C\#} \\
Es un lenguaje de programación basado en objetos desarrollado por Microsoft. En\cite{ortegaAnalisisDisenoImplementacion2017}, los autores mencionan que este lenguaje es uno de los más utilizados, debido a que tiene la característica de poder crear sistemas multiplataforma.
\bigbreak
\textbf{Java} \\
Es un lenguaje de programación orientado a objetos desarrollado por Sun Microsystems. En\cite{montalvoDesarrolloSistemaSoftware2023, ortegaAnalisisDisenoImplementacion2017}, los autores mencionan que este lenguaje es muy utilizado en el desarrollo de aplicaciones web y móviles, puesto que es multiplataforma.

\subsection{Framework}\label{subsec:framework}
Un framework es un conjunto de librerías y herramientas que permiten el desarrollo de aplicaciones de manera más rápida y sencilla.
\bigbreak
\textbf{CodeIgniter} \\
Es un framework de PHP utilizado para el desarrollo web óptimo para aplicaciones que se ejecutan en un hosting compartido y configurados de manera diferente. En\cite{izaDesarrolloERPPara2023} los autores mencionan que este marco de desarrollo agiliza el proceso de desarrollo de aplicaciones web.
\bigbreak

\textbf{ASP.NET} \\
Es un framework de código abierto para el desarrollo de aplicaciones web, el cual es desarrollado por Microsoft. En\cite{ortegaAnalisisDisenoImplementacion2017, moreiraDESARROLLOSISTEMAWEB2019}, los autores mencionan que este framework permite el desarrollo de aplicaciones web de forma rápida y sencilla, debido a que cuenta con una gran cantidad de librerías y componentes que facilitan el desarrollo de aplicaciones web.
\bigbreak
\textbf{Express} \\
Es un framework de código abierto para el desarrollo de aplicaciones web, el cual es desarrollado por la comunidad de Node.js. En\cite{leonardoMejoraControlAsistencia2019}, el autor menciona que este framework brinda un conjunto de características para las aplicaciones web y móviles la cual destacan su función como middleware.
\bigbreak
\textbf{Angular} \\
Es un framework de desarrollo de aplicaciones web creado por Google que permite construir aplicaciones robustas y dinámicas del lado del cliente utilizando TypeScript y una arquitectura basada en componentes.
En\cite{leonardoMejoraControlAsistencia2019}, el autor indica que este framework brinda una alta velocidad y buen rendimiento debido a que convierte las plantillas HTML en código optimizado para JavaScript. En\cite{izaDesarrolloERPPara2023}, mencionan que este framework permite la creación de aplicaciones móviles, ya que posee un lenguaje adaptable.
\bigbreak

\textbf{React Js} \\
React.js es una biblioteca de JavaScript utilizada para construir interfaces de usuario interactivas y dinámicas.
Desarrollada por Facebook, se centra en la creación de componentes reutilizables que representan diferentes partes de la interfaz de usuario.
En\cite{montalvoDesarrolloSistemaSoftware2023}, los autores mencionan que esta librería cuenta con una gran cantidad de librerías construidas por la comunidad, las cuales permiten la creación de aplicaciones web y móviles.
\bigbreak
\textbf{React native} \\
Es un framework de desarrollo de aplicaciones móviles que utiliza JavaScript y React para crear aplicaciones nativas para iOS y Android con un único código base.
En\cite{montalvoDesarrolloSistemaSoftware2023}, los autores mencionan que esta herramienta permite la creación de aplicaciones nativas sin comprometer la experiencia del usuario debido a que brinda un set básico de componentes visuales.

\bigbreak
\textbf{Ionic} \\
Es un framework de desarrollo de aplicaciones móviles híbridas que utiliza tecnologías web como HTML, CSS y JavaScript para crear aplicaciones multiplataforma. En\cite{izaDesarrolloERPPara2023}, los autores indican que esta herramienta optimiza el consumo de recursos para la funcionalidad de una aplicación móvil.
\bigbreak

\textbf{Spring Boot} \\
Es un marco de trabajo o framework de desarrollo de aplicaciones en Java que facilita la creación de aplicaciones robustas de manera rápida y sencilla. En\cite{montalvoDesarrolloSistemaSoftware2023}, los autores mencionan que este marco de desarrollo está basado en el modelo MVC, el cual permite el desarrollo y despliegue de servicios REST.
\bigbreak

\textbf{Retrofit} \\
Es una cliente de servicios REST para Android y Java, el cual es desarrollado por Square.
En\cite{leonardoMejoraControlAsistencia2019}, el autor menciona que esta librería permite realizar peticiones HTTP, gestionar los parámetros y transformar la respuesta en un objeto Java.
\bigbreak
\textbf{RxJava} \\
Es una librería de Java para la programación reactiva basada mediante el uso de observables.
En\cite{leonardoMejoraControlAsistencia2019}, el autor menciona que esta biblioteca brinda una alternativa al uso de Thread y AsyncTask, dado que permite la ejecución de tareas en segundo plano de forma sencilla y asíncronas, además tiene una integración óptima con Retrofit.
\bigbreak
\textbf {Bootstrap} \\
Es una librería de código abierto para el desarrollo de aplicaciones web, la cual permite el desarrollo de aplicaciones web responsivas. En\cite{pintoSistemaGestionAdministrativo2017,ortegaPrototipoSistemaWeb2020}, los autores recalcan que esta librería ofrece componentes que nos permíten interactuar con el usuario al instante, debido a que se encuentran predefinidos.

\subsection{GIT}
Es un sistema de control de versiones distribuido de código abierto, el cual permite el desarrollo de software de forma colaborativa. En\cite{ortegaPrototipoSistemaWeb2020}, el autor menciona que este sistema de control de versiones permite guardar el estado de los archivos de un proyecto en repositorio.

\subsection{Diagramas de diseño}
Un diagrama de diseño es una representación gráfica de un sistema de información, el cual permite visualizar los componentes de un sistema y sus relaciones.
\bigbreak
\textbf{Diagrama de clases} \\
Es un diagrama de estructura estática orientado completamente a la programación orientada a objetos.
En\cite{leonardoMejoraControlAsistencia2019}, los autores mencionan que este diagrama representa la estructura de un sistema, mostrando las clases del sistema, sus atributos y sus relaciones.
\bigbreak
\textbf{ADM-Archimate} \\
Es un lenguaje de modelado de arquitectura empresarial, el cual permite la representación de la arquitectura de una organización. En\cite{nietoPrototipoAplicacionWeb2018}, los autores mencionan que este lenguaje de modelado ofrece los conceptos suficientes para poder modelar una arquitectura empresarial. Sin embargo tambíen indican que no ofrece ninguna metodología para el desarrollo de software.
\bigbreak
\textbf{Ninja Mockup} \\
Es una herramienta de diseño de prototipos de software en línea. En\cite{ortegaPrototipoSistemaWeb2020}, el autor menciona que esta herramienta permite la creación de prototipos de software de forma rápida y sencilla, debido a que cuenta con una gran cantidad de componentes predefinidos.
\bigbreak

\subsection{Patrones de arquitectura de software}
Un patrón de arquitectura de software es un patrón de diseño que permite la creación de una arquitectura de software.
\bigbreak
\textbf{Patrón Modelo-Vista-Controlador(MVC)} \\
Es un patrón de arquitectura de software que separa los datos y la lógica de negocio de una aplicación de la interfaz de usuario y el módulo encargado de gestionar los eventos y las comunicaciones. En\cite{leonardoMejoraControlAsistencia2019}, el autor indica que esta herramienta permite mostrar el acceso a los recursos del sistema gráficamente.
\bigbreak
\textbf{Patrón Model-View-ViewModel(MVVM)} \\
Es un patrón de arquitectura de software que separa los datos y la lógica de negocio de una aplicación de la interfaz de usuario y el módulo encargado de gestionar los eventos y las comunicaciones.
En\cite{leonardoMejoraControlAsistencia2019}, el autor menciona que este patrón de arquitectura es uno de los mejores para el desarrollo móvil debido a que nos permite separar de forma limpia la lógica de presentación y la lógica de negocio.
\bigbreak
\textbf{Representational State Transfer(REST)} \\
Es un estilo de arquitectura de software para sistemas hipermedia distribuidos. En\cite{montalvoDesarrolloSistemaSoftware2023}, los autores mencionan que este estilo de arquitectura permite la comunicación entre sistemas de información de forma sencilla y rápida gracias a los verbos HTTP GET, POST, PUT y DELETE.

\subsection{Metodologías de desarrollo de software}
Una metodología de desarrollo de software es un conjunto de pasos, procedimientos, técnicas y herramientas que permiten el desarrollo de software de forma eficiente.
\bigbreak
\textbf {Metodología RUP} \\
Es una metodología de desarrollo de software iterativa e incremental, la cual tiene como objetivo asegurar la producción de software de alta y mayor calidad. En\cite{moscayzaSistemaWebPara}, los autores mencionan que esta metodología propone mayor interacción con el cliente, por lo tanto, se puede tener un mayor control sobre el proyecto de forma general.
\bigbreak
\textbf{Metodología XP} \\
Es una metodología de desarrollo de software ágil, la cual tiene como objetivo principal la satisfacción del cliente mediante la entrega temprana y continua de software. En\cite{ortegaAnalisisDisenoImplementacion2017}, los autores mencionan que se implementó esta metodología debido al desarrollo rápido en términos de tiempo y ajustes de requisitos a lo largo del desarrollo del mismo que se puedan llevar a cabo.
\bigbreak
\textbf{Metodología SCRUM} \\
Es una metodología ágil que ayuda a los equipos a estructurar y gestionar el trabajo mediante un conjunto de valores, principios y prácticas
En\cite{moreiraDESARROLLOSISTEMAWEB2019}, el autor que esta metodología permite tratar de mejor maneras situaciones imprevisibles y resolver problemas complejos adaptándose a los cambios que se puedan presentar durante el desarrollo del proyecto. En\cite{castroReconocimientoFacialGeolocalizacion2023}, indican que la metodología SCRUM proporciona una vision global del proyecto a desarrollar mediante un cronograma de actividades.

\subsection{Modelo de aceptación tecnológica (TAM)}

Este modelo pretende determinar si una tecnología será aceptada o no por los usuarios basandose en los supuestos de que la percepción de utilidad y facilidad de uso de una tecnología determinan la actitud de un usuario hacia el uso de dicha tecnología. En\cite{tam}, los autores mencionan que este modelo se debe evaluar en dos dimensiones: la percepción de utilidad y la percepción de facilidad de uso. La percepción de utilidad se refiere al grado en que una persona cree que el uso de una tecnología en particular mejorará su desempeño laboral. La percepción de facilidad de uso se refiere al grado en que una persona cree que el uso de una tecnología en particular será libre de esfuerzo.

\subsection{Figma}
Es una herramienta de diseño de interfaces de usuario basada en la nube. En\cite{figma}, los autores mencionan que esta herramienta permite el diseño de interfaces de usuario de forma colaborativa, lo cual permite que varias personas puedan trabajar en un mismo proyecto de forma simultanea. Además, esta herramienta permite la creación de prototipos de software de forma rápida y sencilla, debido a que cuenta con una gran cantidad de plugins.


\bigbreak
De todos los artículos y tesis revisados previamente, se destaca el uso de PHP en la creación de sistemas web junto con la librería de componentes Bootstrap, mientras que para el desarrollo de aplicaciones móviles se tiene como preferencía el desarrollo para sistemas operativos Android mediante el uso de Kotlin como lenguaje de programación. En cuanto a la georreferenciación es indiscutible el uso de GoogleMaps API para obtencion de la referencia geográfica y el uso de los servicios del mismo. En arquitecturas de software destaca MVC debído a que muchos frameworks y libreiras lo implementan nativamente. Por último la métodología de desarrollo de software más utilizada es SCRUM debido a la flexibilidad que ofrece para el desarrollo de proyectos de software.