\subsection{Antecedentes investigativos}
Para el desarrollo de esta investigación es necesario realizar una exploración de trabajos previos relacionados con el problema en los cuales se debe identificar las similitudes y diferencias con el presente proyecto. En este sentido de la búsqueda realizada se obtuvieron los siguientes trabajos previos:
\bigbreak

En\cite{ortegaAnalisisDisenoImplementacion2017} se desarrollaron herramientas tecnológicas web y móvil para la administración de un condominio ubicado en la ciudad de Quito llamado {\textquotedblleft}Torrez de Aranjuez{\textquotedblright}. En dicho estudio se tuvo como objetivo centralizar y optimizar la información para la generación de reportes y mejorar la comunicación entre los usuarios del condominio y los administradores. El sistema web fue desarrollado en el framework .Net usando el lenguaje de programación {C\#} en conjunto con HyperText Markup Language(HTML), Cascading Style Sheets(CSS3) y Javascript. En cuanto a los procesos administrativos cubiertos por el sistema web se encuentran módulos de gestion de pagos, gestión de zonas comunales, reportes finacieros de estados de cuenta, generación de recibos de pago y asistencias de los trabajadores de mantenimiento del conjunto. Mientras que el sistema móvil fue desarrolado unicamente para dispositivos Android utilizando el lenguaje de programación Java y cuenta con los módulos de gestion de pagos y notificaciones. Como metodología para el desarrollo del sistema se utilizó Extreme Programming (XP) debido a la flexibilidad de dicha métodología para el manejo de roles dentro de el ciclo de vida del proyecto y se usó como gestor de bases de datos Microsfot SQL Server.
\bigbreak
En\cite{pintoSistemaGestionAdministrativo2017} se implementó un sistema de gestión administrativa para el condominio {\textquotedblleft}Puertas de Alcalá{\textquotedblright} cuyo objetivo fue crear una plataforma informativa para los residentes de manera que toda la información sea transparente centrada específicamente en el control de pagos. En este sentido el sistema web fue desarrollado en Hypertext Preprocessor(PHP), JavaScript y SQL Server como base de datos sin el uso de ningun tipo de framework. El sistema unicamente cuenta con procesos de gestión de pagos y reportes de pagos. La metodología utilizada fue la metodología de desarrollo de software iterativa.
\bigbreak
En\cite{nietoPrototipoAplicacionWeb2018} se propone un prototipo de modelado de software utilizando el protocolo de intercambio de información Simple Object Access Protocol (SOAP). En este estudio se tomó como población a
129 residentes a los cuales se les aplicó una encuesta inicial para determinar las necesidades a cumplir mínimas por un sistema de votaciones de asambleas comunales.
En donde se pudo evidenciar que las asambleas tienen una duración excesiva y se tiene la necesidad de optar por votaciones electrónicas debido a que el sistema manual presenta deficiencias en términos de transparencia.
En este sentido el prototipo propuesto sugiere que se implementen roles a los usuarios para el manejo de seguridad, un módulo para la gestión de contabilidad, un módulo para la gestion legal, un módulo para la gestión de las asambleas y un módulo para la administración de los bienes comunales. La métodología utilizada para este caso de estudio fue Rational Unified Process(RUP).
\bigbreak
En\cite{moscayzaSistemaWebPara} se desarrolló un sistema web para la gestión de presupuestos en el Edificio Condominio Aquamar en Peru ubicado en el distrito La perla, Callao. En dicho estudio de tipo experimental se contó con una poblacion de 204 residentes y se utilizó un enfoque cuantitativo para la recolección de datos. El objetivo principal del estudio fue determinar la influencia del sistema web a los procesos de gestion del presupuesto. Teniendo en cuenta los indicadores de indice de ingresos de montos corrientes, liquidez y tiempo promedio de respuesta, con los cuales se busco analizar los efectos del sistema en los procesos. En este sentido el sistema propuesto fue desarrollado en PHP y MySQL sin el uso de ningún framework de desarrollo web, y se utilizó la metodología RUP. El sistema unicamente cuenta con módulos relacionados a la gestion econónomica más no a otros procesos administratrativos tales como incidentes, buzón de sugerencia, zonas comunales, parqueaderos y asambleas. Los resultados obtenidos fueron que el sistema web propuesto permitió mejorar los procesos de gestión del presupuesto en el Edificio Condominio Aquamar.
\bigbreak
En\cite{leonardoMejoraControlAsistencia2019} se desarrollaron sistemas para el control de asistencias de personal a través de reconocimiento facial y geolocalización. En este estudio se desarrollo una aplicación móvil llamada SICAP cuya finalidad es la de registrar la asistencia de los empleados mediante el reconocimiento facil y su ubicación en tiempo real en un area de 50 metros de cualquiera de las areas de trabajo designadas por la empresa Agro Rural. Cabe mencionar que esta aplicación no requiere de internet para su uso puesto que utiliza un paradagima de programación reactiva lo cual permite que la aplicación envie sus peticiones cuando el dispositivo tenga acceso a internet. La Interfaz de Programación de Aplicaciones (API) desarrollada en este estudio permite exponer el servicio de asistencia a cualquier otro sistema desarrollado por la empresa Agro Rural. Por otro lado la aplicación web tiene la finalidad de la visualización de las asistencias por parte de los empleados, gestionar las incidencias laborales y generar reportes. Para el desarrollo de la aplicación móvil se utilizó el lenguaje de programación Kotlin junto con las librerias RxJava, EventBus y Retrofit. Mientras que para la aplicacion web se utilizo el framework de desarrollo Angular y para la API se utilizo el framework de JavaScript Express.
\bigbreak
En\cite{moreiraDESARROLLOSISTEMAWEB2019} se desarrolló un sistema web alojado en la nube de Microsoft Azure para mejorar el control de los procesos administrativos y financieros del condominio {\textquotedblleft}Solar del Rio{\textquotedblright}.En dicho estudio se contó con una población de 82 propietarios de viviendas. Para el desarrollo del sistema se utilizó el framework .Net Core y SQL Server como gestor de bases de datos. En cuanto a la metodología de desarollo se utilizó SCRUM la cual permitión tener un mejor control de los procesos a seguir durante el desarollo del sistema. Adicionalmente se aplicó la norma ISO/IEC 25010 para la evaluación de la calidad del sistema. El sistema cuenta con módulos de gestión de pagos, gestión de parqueaderos, impresión de recivos de pago y reportes de pagos. Para la validación del sistema se utilizó el Cuestionario de usabilidad de sistemas informáticos (CSUQ) y el métodos estadisticos de 2 variables para obtener la relación entre las preguntas del la encuesta.
\bigbreak
En\cite{lopezPROTOTIPOSOFTWAREWEB2020} se desarrolló un prototipo de software para la gestión de votaciones en las asambleas de condominios. En este estudio el prototipo desarrollado cuenta con un módulo de seguridad para el registro de usuarios y permisos. Ademas de el módulo de votaciones en el cual se pueden colocar observaciones adicionales al voto. Para el desarrollo de este prototipo se utilizo el framewwork de JavaScript Node.js junto con el motor de bases de datos SQL Server. La métodología utilizada fue Xtreme Programming (XP). Para la toma de información se utilizaron dos cuestionarios en 10 conjuntos residenciales, el primero a los usuarios residentes y el segundo a los administradores de los condominios. En dichos cuestionarios se evidencio la necesidad de tener un sistema informático para realizar las votaciones en asambleas comunales.
\bigbreak
En\cite{ortegaPrototipoSistemaWeb2020} se implementó un sistema web para optimizar la gestión de actividades y eventos en el condominio {\textquotedblleft}Los Jardines{\textquotedblright}. En este estudio se tuvo una poblacion de 266 residentes. El sistema cuenta con las funciones de poder registrar eventos o actividades comunales, con un módulo de buzon de sugerencias que a su vez funciona como un foro de objetos perdidos y un módulo de reservas de espacios comunales. El sistema no cuenta con funcionalidades relacionadas con la gestion financiera de un conjunto residencial. Para desarrollar este sistema web se utilizó PHP, JavaScript, CSS, HTML y Bootstrap. La metodología utilizada fue la metodología de desarrollo de software en cascada y como gestor de bases de datos se uso MYSQL.
\bigbreak
En\cite{portugalSistemaInformacionPara2022} se desarrolló un sistema de información web para el condominio {\textquotedblleft}jardines de Aramburú{\textquotedblright} ubicado en Perú. En dicho estudio se tuvo como poblacion a 550 propietarios, de los cuales se tomaron dos muestras. La primera muestra fueroin los 3 miembros de la junta de propietarios y la segunda 100 propietarios seleccionados de manera aleatoria. La investigación se centro en los procesos de gestión administrativa tales como pagos, mantenimientos de zonas comunales, y reportes financieros. para el levantamiento de información se aplicaron encuetas a los 100 propietarios lo cual les permitió medir de forma cuantitativa y estadisticamente. El sistema web se desarrollo en PHP y MySQL sin el uso de ningún framework de desarrollo web.
\bigbreak
En\cite{castroReconocimientoFacialGeolocalizacion2023} se implementó un sistema para mejorar los procesos de asistencia mediante la aplicación de geolocalización y reconocimiento facial. Para este estudio de investigación se utilizó un enfoque cuantitativo de diseño pre-experimental. En el sistema se implemento un modulo de administración de empleados en donde se registra la información personal y una foto del empleado para validar el reconocimiento facial. Para validar la asistencia tambien se colocan las coordenadas geograficas de las sucursales de la empresa en donde mediante el uso del Sistema de Posicionamiento Global(GPS) del dispositivo móvil se valida la asistencia del empleado. El sistema utiliza la herramienta de desarrollador de Google Maps API.
\bigbreak
En\cite{izaDesarrolloERPPara2023} se desarrolló una aplicación web y móvil para el registro de asistencia con geolocalización de los empleados de la empresa proveedora de Internet SISCOM en Latacunga. En este estudio se utilizaron encuestas y entrevistas para el levantamiento de información e identificar las necesidades. En este sentido para la aplicación web se utilizó Laravel 7 como framework de desarrollo. Mientras que para la aplicación móvil se utilizó el framework de desarrollo IONIC basado en Angula. Las métodolgías empleadas son XP y Mobil-D respectivamente. Como resultado de este proyecto los autores indican que se pudo optimizar el control de asistencia de los empleados asi como la veracidad de la misma.
\bigbreak
En\cite{montalvoDesarrolloSistemaSoftware2023} se desarrolló un software multiplataforma para la optimización del modelo de gestión que posee el condominio {\textquotedblleft}Conjunto habitacional oriental{\textquotedblright}. En este estudio se demostró que el sistema con un nivel de confianza del 95\% y un margen de error del 5\% optimizó el modelo de gestión. En este proyecto se utilizó una base de datos PostgreSQL junto con los frameworks Spring Boot, React.js y React Native. La capa de negocio se enmarcó en la arquitectura REST. Por último la metodlogía utilizada fue SCRUM.