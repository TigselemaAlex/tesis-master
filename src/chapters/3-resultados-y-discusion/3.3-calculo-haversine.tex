\section{Cálculo de la distancia Haversine}\label{sec:calculo-de-la-distancia-haversine}

El cálculo de la distancia entre dos puntos geográficos es una tarea común en la programación de aplicaciones que requieren la ubicación de un usuario o la ubicación de un lugar específico.
En el caso de la aplicación móvil, se requiere conocer si el usuario se encuentra dentro de un rango de distancia de el lugar de reuniones en las asambleas de conjunto residencial.
\bigbreak
En \cite{camino_costa_desarrollo_2012} el autor menciona que la fórmula de Haversine es una fórmula utilizada para calcular la distancia entre dos puntos en la superficie de una esfera, en este caso, la Tierra.
Además, menciona que esta fórmula es precisa en el cálculo numérico incluso a distancias pequeñas, lo cual es ideal debido a que la precisión es un factor importante en la aplicación móvil, ya que será determinante para la verificación de la ubicación de los residentes.
\bigbreak
La fórmula de Haversine se define como:


\begin{itemize}
  \item $R$: Radio de la Tierra (aproximadamente 6371 km o 6371e3 metros).
  \item $\text{lat1}$, $\text{lon1}$: Latitud y longitud del primer punto en grados.
  \item $\text{lat2}$, $\text{lon2}$: Latitud y longitud del segundo punto en grados.
  \item $dLat$: Diferencia de latitud en radianes.
  \item $dLon$: Diferencia de longitud en radianes.
  \item $a$: Fórmula de Haversine.
  \item $c$: Ángulo central en radianes.
  \item $d$: Distancia entre los dos puntos en metros.
\end{itemize}


\begin{equation}
    \begin{aligned}
        \text{R} &= 6371e3 \\
        \text{dLat} &= \text{lat2} - \text{lat1} \\
        \text{dLon} &= \text{lon2} - \text{lon1} \\
        a &= \sin^2\left(\frac{dLat}{2}\right) + \cos(\text{lat1}) \cdot \cos(\text{lat2}) \cdot \sin^2\left(\frac{dLon}{2}\right) \\
        c &= 2 \cdot \text{atan2}\left(\sqrt{a}, \sqrt{1-a}\right) \\
        d &= R \cdot c \\
    \end{aligned}\label{eq:equation3}
\end{equation}





