\section{Definición de la metodología de desarrollo de software}\label{sec:definicion-metodologia}

En la siguiente sección se detalla la metodología de desarrollo de software que se utilizará para la implementación del sistema de gestión de conjunto habitacional.
La metodología de desarrollo de software es un enfoque estructurado para la creación de sistemas de software.
Existen dos tipos de metodologías de desarrollo de software: metodologías tradicionales y metodologías ágiles.
A continuación, se mostrará una comparación entre ambas metodologías y se justificará la elección de la metodología seleccionada para el desarrollo del sistema de gestión de conjunto habitacional.

\subsection{Comparación entre metodologías tradicionales y metodologías ágiles}\label{subsec:comparacion-entre-metodologias-tradicionales-y-metodologias-agiles}

En la tabla \ref{tab:table_agil_vs_tradicional} se muestra una comparación entre las metodologías ágiles y las metodologías tradicionales.
En donde se puede evidenciar que las metodologías ágiles se enfocan más en la adaptación de posibles cambios, en proyectos pequeños y con un equipo de trabajo pequeño junto con una buena colaboración entre el equipo de desarrollo y el cliente.
Por otro lado las metodologías tradicionales se evidencia su utilidad de aplicación en proyectos grandes, con un equipo de trabajo grande y con una planificación previa amplia.
Por el análisis previo se llegó a la conclusión que para llevar este proyecto a cabo se utilizará una metodología ágil, ya que se cuenta con un equipo de trabajo pequeño, con entregas continuas al cliente y con la posibilidad de adaptarse a los cambios que se presenten durante el desarrollo del sistema.
En \cite{islam_comparison_2020} se extrajo la siguiente tabla que compara las metodologías ágiles con las metodologías tradicionales.

\begin{table}[H]
    \centering
    \caption[Comparación entre metodologías ágiles y metodologías tradicionales]{Comparación entre metodologías ágiles y metodologías tradicionales traducido de \cite{islam_comparison_2020} }\label{tab:table_agil_vs_tradicional}
    \renewcommand*{\arraystretch}{1.4}
    \begin{footnotesize}
    \begin{tabular}{ |>{\bfseries}l|l|l| }
        \hline
        \multicolumn{1}{|c|}{ \textbf{Característica}} & \multicolumn{1}{c|}{\textbf{Metodologías Ágiles}} & \multicolumn{1}{c|}{ \textbf{Metodologías Tradicionales}} \\
        \hline
        Enfoque                                      & Adaptativo                                       & Predictivo                                               \\
        \hline
        Medición de éxito                            & Valor de negocio                                 & Conforme al plan                                         \\
        \hline
        Tamaño
        del proyecto                                & Pequeño                                           & Grande                                                   \\
        \hline
        Estilo de gestión                            & Descentralizado                                  & Autocrático                                              \\
        \hline
        Perspectiva de cambio                        & Adaptable                                        & Sostenible                                 \\
        \hline
        Cultura                                      & Liderazgo - Colaboración                         & Ordenar - Controlar                                      \\
        \hline
        Documentación                                & Baja                                             & Alta                                                     \\
        \hline
        Énfasis                                      & Orientado al cliente                             & Orientado al proceso                                     \\
        \hline
        Ciclos                                       & Numerosos                                        & Limitados                                                \\
        \hline
        Domínio                                      & Impredecible/Exploratorio                        & Predecible                                               \\
        \hline
        Planificación previa                         & Mínima                                           & Amplia                                                   \\
        \hline
        Retorno de la inversión                      & Principio del proyecto                           & Fin del proyecto                                         \\
        \hline
        Tamaño del equipo                            & Pequeño                                          & Grande                                                   \\
        \hline
    \end{tabular}
    \end{footnotesize}
\end{table}

Por otro lado para escoger la metodología ágil que se utilizará en el desarrollo del sistema de gestión de conjunto habitacional se realizó un análisis de las metodologías ágiles más utilizadas en la actualidad detalladas a continuación en la siguiente tabla.
En \cite{barriga_sanchez_sistema_2023} se extrajo la siguiente tabla que compara algunas de las metodologías ágiles más utilizadas en la actualidad.

\newpage
\begin{footnotesize}
\begin{spacing}{1}
    \begin{center}
        \renewcommand*{\arraystretch}{1.4}
        \begin{longtable}[l]{ |>{\bfseries}p{0.15\textwidth}| p{0.17\textwidth} |p{0.17\textwidth}  |p{0.17\textwidth} | p{0.17\textwidth}|}
            \caption[Comparación de las metodolgías de desarrollo ágiles]{ Comparación de las metodolgías de desarrollo ágiles\cite{barriga_sanchez_sistema_2023} }\label{tab:table_metodologias_agiles} \\
            \hline
            \multicolumn{1}{|c|}{ \textbf{Criterio}} & \multicolumn{1}{c|}{\textbf{XP}}                                                                        & \multicolumn{1}{c|}{ \textbf{Lean}} & \multicolumn{1}{c|}{ \textbf{RAD}}  & \multicolumn{1}{c|}{ \textbf{Kanban}}\\
            \hline
            Enfoque                                & Iterativo e incremental                                                                                & Iterativo e incremental                                                                        & Prototipado                                                                                & Continuo                                                                                \\
            \hline
            Principios                             & Integración continua, programación en pares, desarrollo basado en pruebas, comentarios de los clientes & Centrarse en el valor, eliminar desperdicios, flujo, mejora continua, respeto por las personas & Desarrollo rápido, participación del usuario, desarrollo iterativo, creación de prototipos & Visualización del flujo de trabajo, limitación del trabajo en progreso, mejora continua \\
            \hline
            Tamaño del equipo                      & 3--5                                                                                                   & 2--3                                                                                           & 2--3                                                                                       & No definido                                                                             \\
            \hline
            Tamaño del proyecto                    & Pequeños y medianos con requisitos bien definidos                                                      & Efectivo para proyectos grandes con requisitos complejos & Pequeños y medianos con requisitos cambiantes & Pequeños, medianos y grandes \\
            \hline
            Ventajas & Calidad y comunicación & Velocidad y flexibilidad & Costo y tiempo & Mejor flujo de trabajo \\
            \hline
            Simplicidad                            & Se busca la simplicidad en el código, en el diseño y en la solución de problemas & Se busca eliminar desperdicio y complejidad innecesaria para mejorar la eficiencia & Se enfoca en desarrollar soluciones simples y rápidas, evitando excesos de diseño & Busca eliminar desperdicio, simplificar procesos y claridad de flujo de trabajo \\
            \hline
            Entrega de software                    & Frecuente y regular                                                                                    & Entrega incremental                                                                            & Entrega rápida                                                                             & Entrega continua                                                                        \\
            \hline
            Planificación                          & Planificación continua                                                                                 & Planificación y modelado                                                                       & Planificación rápida y flexible & Planificación continua y visual \\
            \hline
        \end{longtable}

    \end{center}
\end{spacing}
\end{footnotesize}

Del análisis de la Tabla \ref{tab:table_metodologias_agiles} se llegó a la conclusión de que la metodología ágil que se utilizará en el desarrollo del proyecto será RAD, ya que se ajusta a las necesidades del proyecto, debido a que el equipo de desarrollo es pequeño, el proyecto es de tamaño mediano y se cuenta con tiempo limitado para la entrega del sistema.
Además, sus características de desarrollo rápido junto con la entrega de prototipos funcionales y participación del usuario, permitirá una mayor adaptabilidad a los cambios que se presenten durante el desarrollo del sistema.
