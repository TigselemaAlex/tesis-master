\section{Desarrollo de la propuesta}\label{sec:desarrollo-propuesta}

Para el desarrollo de la propuesta se aplicó la metodología RAD la cual según\cite{bonilla_cadena_desarrollo_2022}, la autora menciona que esta metodología ayuda al desarrollo rápido de aplicaciones de una manera rápida y económica enfocada principalmente a empresas con baja disponibilidad de recursos y tiempo.

La metodología RAD se basa en cuatro fases principales:

\begin{itemize}
    \item Planificación de requerimientos \\
    En esta primera fase se identifican los requerimientos del sistema para satisfacer las necesidades del cliente y se establece el alcance del proyecto.
    \item Diseño de usuario\\
    Una vez identificados los requerimientos se crean modelos de diseño previos a la construcción del sistema los cuales son presentados a los usuarios para recibir retroalimentación.
    \item Construcción\\
    En esta fase se construye el sistema de acuerdo a los modelos de diseño previamente aprobados, mediante la codificación y pruebas del sistema.
    \item Transición
    La fase final en la que se realiza la entrega del sistema levantado en un entorno de producción real en donde se realizarán pruebas finales y se capacitará a los usuarios finales.
\end{itemize}

\subsection{Planificación de requerimientos} \label{subsec:planificacion-requerimientos}

En esta primera etapa se realizó un análisis de la información recopilada en las entrevistas, encuestas y en la situación actual descrita anteriormente.

En las siguientes tablas continuación se detallan en los usuarios identificados en la interacción con el sistema web y la aplicación móvil.
En la Tabla \ref{tab:table_requerimientos} se detallan los requerimientos del proyecto identificados por usuarios.

\begin{table}[H]
    \centering
    \caption{Descripción de los usuarios identificados en la interacción con el sistema web administrativo}
    \renewcommand*{\arraystretch}{1.4}
    \begin{longtable}{|p{0.2\textwidth}|p{0.7\textwidth}|}
        \hline
        \textbf{Usuario} & \textbf{Descripción}                                                                           \\
        \hline
        Administrador    & Usuario encargado de la administración de usuarios y la configuración de la geolocalización.   \\
        \hline
        Presidente       & Usuario encargado de la administración de parqueaderos, residencias, guardianía, convocatorias \\
        \hline
        Vicepresidente   & Usuario encargado de la administración de residencias, guardianía y convocatorias              \\
        \hline
        Tesorero         & Usuario encargado de la administración de los ingresos, multas y egresos                       \\
        \hline
        Secretario       & Usuario encargado de la administración de los eventos sociales y convocatorias                 \\
        \hline
    \end{longtable}\label{tab:table_usuarios_web}
\end{table}

\begin{table}[H]
    \centering
    \caption{Descripción de los usuarios identificados en la interacción con la aplicación móvil}
    \renewcommand*{\arraystretch}{1.4}
    \begin{longtable}{|p{0.2\textwidth}|p{0.7\textwidth}|}
        \hline
        \textbf{Usuario} & \textbf{Descripción}                                                                                                                                                 \\
        \hline
        Propietario      & Usuario encargado de registrar la asistencia en asambleas y votación, visualización de las obligaciones financieras relacionadas con su residencia o sus residencias \\
        \hline
        Inquilino        & Usuario encargado de registrar la asistencia en asambleas y visualización de las obligaciones financieras relacionadas con su residencia o sus residencias           \\
        \hline
    \end{longtable}\label{tab:table_usuarios_movil}
\end{table}

\begin{spacing}{1}

    \begin{center}
        \renewcommand*{\arraystretch}{1.4}
        \begin{longtable}[l]{|p{0.045\textwidth}|p{0.19\textwidth}|p{0.4\textwidth}|p{0.13\textwidth}| p{0.09\textwidth}|}
            \caption{Identificación de los requerimientos del sistema} \\
            \hline
            \textbf{ID} & \textbf{Requerimiento}                            & \textbf{Descripción}                                                                                                                                                                                                                                                                                                                                                                                                                                                                   & \textbf{Prioridad}         & \textbf{Riesgo}           \\
            \hline
            \multicolumn{5}{|l|}{ \textbf{Todos los usuarios} } \\
            \hline
            R1          & Iniciar sesión                                    & El usuario podrá iniciar sesión mediante su autenticación ingresando sus credenciales: cédula y contraseña & \multicolumn{1}{c|}{Alta} & \multicolumn{1}{c|}{Alto}\\
            \hline
            R2          & Cerrar sesión                                     & El usuario podrá finalizar la sesión en cualquier momento                                                                                                                                                                                                                                                                                                                                                                                                                              & \multicolumn{1}{c|}{Alta} & \multicolumn{1}{c|}{Bajo}\\
            \hline
            R3          & Recuperar contraseña                              & El usuario en caso de olvidar la contraseña podrá solicitar una contraseña nueva autogenerada por el sistema y enviada a su correo electrónico & \multicolumn{1}{c|}{Media} & \multicolumn{1}{c|}{Bajo}\\
            \hline
            R4          & Cambiar contraseña                                & El usuario podrá cambiar su contraseña en cualquier momento & \multicolumn{1}{c|}{Media} & \multicolumn{1}{c|}{Bajo}\\
            \hline
            \multicolumn{5}{|l|}{ \textbf{Administrador} } \\
            \hline
            R5          & Gestionar usuarios                                & El administrador podrá visualizar, registrar, inhabilitar o editar la información y roles de los demás usuarios  & \multicolumn{1}{c|}{Alta} & \multicolumn{1}{c|}{Alto}\\
            \hline
            R6          & Gestionar roles                                   & El administrador podrá visualizar o editar la descripción de los roles existentes en el sistema  & \multicolumn{1}{c|}{Media} & \multicolumn{1}{c|}{Bajo}\\
            \hline
            R7          & Gestionar pasajes                                 & El administrador podrá visualizar o editar la descripción de los pasajes existentes en el sistema  & \multicolumn{1}{c|}{Media} & \multicolumn{1}{c|}{Bajo}\\
            \hline
            R8          & Gestionar geolocalización                         & El administrador podrá editar la ubicación de las coordenadas y el radio de aceptación mediante la visualización de un mapa del lugar en donde se darán lugar las asambleas  & \multicolumn{1}{c|}{Alta} & \multicolumn{1}{c|}{Alto}\\
            \hline
            \multicolumn{5}{|l|}{ \textbf{Presidente} } \\
            \hline
            R9 & Gestionar estacionamientos & El presidente podrá visualizar, eliminar o actualizar la residencia asociada a cada parqueadero.
            También podrá visualizar o editar la descripción de los tipos de parqueaderos existentes & \multicolumn{1}{c|}{Alta} & \multicolumn{1}{c|}{Alto}\\
            \hline
            \multicolumn{5}{|l|}{ \textbf{Presidente y Vicepresidente} } \\
            \hline
            R10         & Gestionar residencias                             & El presidente o vicepresidente podrán visualizar, eliminar o actualizar el inquilino o propietario de cada residencia del conjunto & \multicolumn{1}{c|}{Alta} & \multicolumn{1}{c|}{Alto}\\
            \hline
            R11         & Gestionar guardias                                & El presidente o vicepresidente podrán visualizar, registrar, editar, o inhabilitar a los guardias de seguridad  & \multicolumn{1}{c|}{Alta} & \multicolumn{1}{c|}{Alto}\\
            \hline
            R12 & Gestionar actividades de guardianía & El presidente o vicepresidente podrá visualizar, registrar, editar, o eliminar las actividades de guardianía.
            También podrá cambiar el estado de cada actividad & \multicolumn{1}{c|}{Alta} & \multicolumn{1}{c|}{Alto}\\
            \hline
            R13         & Gestionar los tipos de incidentes                 & El presidente o vicepresidente podrán visualizar, registrar, editar, o inhabilitar los tipos de incidentes & \multicolumn{1}{c|}{Alta} & \multicolumn{1}{c|}{Alto}\\
            \hline
            R14         & Gestionar incidentes                              & El presidente o vicepresidente podrán visualizar, crear, editar, o eliminar los incidentes reportados por los guardías, asi como el cambio del estado del incidente & \multicolumn{1}{c|}{Alta} & \multicolumn{1}{c|}{Alto}\\
            \hline
            R15         & Gestionar convocatorias                           & El presidente o vicepresidente podrá visualizar, descargar, registrar, editar, finalizar o eliminar las convocatorias,  & \multicolumn{1}{c|}{Alta} & \multicolumn{1}{c|}{Alto}\\
            \hline
            R16         & Gestionar asistencias de las asambleas            & Las convocatorias de tipo asamblea son las únicas que poseerán registro de asistencias, de tal manera que el presidente o vicepresidente podrán realizar el registro manual de las asistencias, asi como descargar un reporte de inasistentes & \multicolumn{1}{c|}{Alta} & \multicolumn{1}{c|}{Alto}\\
            \hline
            R17         & Gestionar votaciones de las asambleas             & Las convocatorias de tipo asamblea son las únicas que poseerán votaciones, de tal manera que el presidente o vicepresidente podrán visualizar, editar, habilitar el voto o eliminar de cada pregunta propuesta a votación, asi como también poder visualizar la información de votación de cada votante & \multicolumn{1}{c|}{Alta} & \multicolumn{1}{c|}{Alto}\\
            \hline
            \multicolumn{5}{|l|}{ \textbf{Tesorero} } \\
            \hline
            R18         & Gestionar ingresos mensuales                      & Posterior a la entrega física o digital del comprobante de pago por parte del residente la tesorera procede a registrar los datos del pago junto con el comprobante teniendo en cuenta el último mes de pago y hasta que mes está abonando el residente y posteriormente se envía al correo electrónico un respaldo del registro del pago, adicional a esto también puede visualizar, editar la información así como actualizar el comprobante de pago o eliminar el registro del pago & \multicolumn{1}{c|}{Alta} & \multicolumn{1}{c|}{Alto}\\
            \hline
            R19         & Gestionar ingresos casuales                       & Posterior a la entrega física o digital del comprobante de pago por parte del residente la tesorera procede a registrar los datos del pago junto con el comprobante y posteriormente se envía al correo electrónico un respaldo del registro del pago, adicional a esto también puede visualizar, editar la información así como actualizar el comprobante de pago o eliminar el registro del pago & \multicolumn{1}{c|}{Alta} & \multicolumn{1}{c|}{Alto}\\
            \hline
            R20 & Gestionar multas & La tesorera podrá visualizar, crear, editar o eliminar las multas.
            Los residentes deben presentar de manera física o digital el comprobante del pago de la multa por el monto indicado y posterior a su revisión se procede a subir el comprobante y a actualizar el estado del pago de la multa, posteriormente se envía al correo electrónico un respaldo del registro del pago & \multicolumn{1}{c|}{Alta}  & \multicolumn{1}{c|}{Alto}\\
            \hline
            \multicolumn{5}{|l|}{ \textbf{Secretario} } \\
            \hline
            R21         & Gestionar eventos sociales                        & El secretario podrá visualizar, registrar, editar, o eliminar los eventos sociales. & \multicolumn{1}{c|}{Media} & \multicolumn{1}{c|}{Bajo}\\
            \hline
            R22         & Subir actas                           & El secretario podrá subir actas de las convocatorias & \multicolumn{1}{c|}{Media} & \multicolumn{1}{c|}{Bajo}\\
            \hline
            \multicolumn{5}{|l|}{ \textbf{Propietario e Inquilino} } \\
            \hline
            R23         & Visualizar el calendario                          & El propietario o inquilino podrán visualizar el calendario de eventos sociales y asambleas próximas & \multicolumn{1}{c|}{Media} & \multicolumn{1}{c|}{Bajo}\\
            \hline
            R24         & Visualizar estado de las obligaciones financieras & El propietario o inquilino podrán visualizar el estado financiero de sus obligaciones financieras de todas sus residencias o parqueaderos & \multicolumn{1}{c|}{Media} & \multicolumn{1}{c|}{Bajo}\\
            \hline
            R25         & Visualizar asamblea del día                       & El propietario o inquilino podrán visualizar la asamblea únicamente si se da en ese día & \multicolumn{1}{c|}{Alta} & \multicolumn{1}{c|}{Bajo}\\
            \hline
            R26         & Registrar asistencia                              & El propietario o inquilino podrán registrar su asistencia siempre que se encuentre dentro del rango de geolocalización registrado por el administrador del sistema & \multicolumn{1}{c|}{Alta} & \multicolumn{1}{c|}{Alto}\\
            \hline
            \multicolumn{5}{|l|}{ \textbf{Propietario} } \\
            \hline
            R27         & Registrar voto                                    & El propietario podrá registrar su voto en las preguntas que se encuentren habilitadas para su voto siempre que se encuentre dentro del rango de geolocalización registrado por el administrador del sistema y tenga registrada la asistencia& \multicolumn{1}{c|}{Alta} & \multicolumn{1}{c|}{Alto}\\
            \hline
        \end{longtable}\label{tab:table_requerimientos}
    \end{center}
\end{spacing}

Una vez definidos los requerimientos del sistema se define el plan de trabajo en cada iteración para el desarrollo de los sistemas.

\begin{spacing}{1}
    \begin{center}
        \renewcommand*{\arraystretch}{1.4}
        \begin{longtable}[l]{|p{0.17\textwidth}|p{0.045\textwidth}|p{0.055\textwidth}|p{0.37\textwidth}| p{0.09\textwidth}|p{0.09\textwidth}| }
            \caption{Identificación de los requerimientos del sistema} \\
            \hline
            \multirow{2}{*}{\textbf{N° Iteración}} & \multirow{2}{*}{\textbf{N°}} & \multirow{2}{*}{\textbf{ID}} & \multirow{2}{*}{\textbf{Requerimiento}} & \multicolumn{2}{c|}{\textbf{Tiempo estimado}}\\
            \cline{5-6}
            & & & & \textbf{Horas} & \textbf{dias} \\
            \hline
            \multirow{8}{*}{Iteración 1} & 1 & R1 & Iniciar sesión & 6 & 1 \\
            \cline{2-6}
             & 2 & R2 & Cerrar sesión & 1 & 1 \\
            \cline{2-6}
             & 3 & R3 & Recuperar contraseña & 2 & 1 \\
            \cline{2-6}
             & 4 & R4 & Cambiar contraseña & 1 & 1 \\
            \cline{2-6}
             & 5 & R5 & Gestionar usuarios & 10 & 2 \\
            \cline{2-6}
             & 6 & R6 & Gestionar roles & 3 & 1 \\
            \cline{2-6}
             & 7 & R7 & Gestionar pasajes & 3 & 1 \\
            \cline{2-6}
             & 8 & R8 & Gestionar geolocalización & 6 & 1 \\
            \hline
            \multirow{9}{*}{Iteración 2} & 9 & R9 & Gestionar estacionamientos & 16 & 2 \\
            \cline{2-6}
             & 10 & R10 & Gestionar residencias & 8 & 1 \\
            \cline{2-6}
             & 11 & R11 & Gestionar guardias & 6 & 1 \\
            \cline{2-6}
             & 12 & R12 & Gestionar actividades de guardianía & 8 & 1 \\
            \cline{2-6}
             & 13 & R13 & Gestionar los tipos de incidentes & 4 & 1 \\
            \hline
             & 14 & R14 & Gestionar incidentes & 8 & 1 \\
            \cline{2-6}
             & 15 & R15 & Gestionar convocatorias & 16 & 2 \\
            \cline{2-6}
             & 16 & R16 & Gestionar asistencias de las asambleas & 8 & 1 \\
            \cline{2-6}
             & 17 & R17 & Gestionar votaciones de las asambleas & 10 & 2 \\
            \hline
            \multirow{10}{*}{Iteración 3} & 18 & R18 & Gestionar ingresos mensuales & 16 & 2 \\
            \cline{2-6}
             & 19 & R19 & Gestionar ingresos casuales & 8 & 1 \\
            \cline{2-6}
             & 20 & R20 & Gestionar multas & 14 & 2 \\
            \cline{2-6}
             & 21 & R21 & Gestionar eventos sociales & 8 & 1 \\
            \cline{2-6}
             & 22 & R22 & Subir actas & 2 & 1 \\
            \cline{2-6}
             & 23 & R23 & Visualizar el calendario & 4 & 1 \\
            \cline{2-6}
             & 24 & R24 & Visualizar estado de las obligaciones financieras & 12 & 2 \\
            \cline{2-6}
             & 25 & R25 & Visualizar asamblea del día & 4 & 1 \\
            \cline{2-6}
             & 26 & R26 & Registrar asistencia & 8 & 1 \\
            \cline{2-6}
             & 27 & R27 & Registrar voto & 8 & 1 \\
            \hline
        \end{longtable}
    \end{center}
\end{spacing}

\subsection{Diseño de usuario} \label{subsec:diseno-usuario}
\subsubsection{Análisis de los procesos sistematizados propuestos}

A continuación se detallaran los procesos sistematizados para la administración de parqueaderos, convocatorias y administración financiera.

\begin{itemize}
    \item Proceso de administración de parqueaderos (Zona azul).
    \begin{enumerate}
        \item Se verifica que el propietario o inquilino este registrado en el sistema.
        \begin{enumerate}
            \item Si está registrado se procede al siguiente proceso.
            \item Si no está registrado se le notifica al administrador para su registro en el sistema y se repite el proceso.
        \end{enumerate}
        \item El presidente solicita enviar la documentación requerida para la asignación de un parqueadero.
        \begin{enumerate}
            \item Si entrega la documentación completa se procede al siguiente proceso.
            \item Si no entrega la documentación completa se finaliza el proceso.
        \end{enumerate}
        \item El presidente muestra en el sistema el mapa de parqueaderos disponibles.
        \item El propietario o inquilino debe abonar diez dólares por la asignación del parqueadero y enviar el comprobante de pago a la tesorera.
        \item La tesorera verifica el comprobante de pago de diez dólares por la asignación del parqueadero.
        \item Se registra el pago en los ingresos mensuales en el sistema.
        \item Se sube el comprobante de pago al sistema.
        \item Se envía por correo electrónico el respaldo de pago al propietario o inquilino.
        \item El presidente asigna el parqueadero al domicilio relacionado al propietario o inquilino.
        \item El sistema genera una carta de compromiso.
    \end{enumerate}

    \item Proceso de administración de convocatorias.
    \begin{enumerate}
        \item La directiva del conjunto se reúne para definir la fecha de la convocatoria.
        \item Se registra en el sistema la convocatoria y se notifica a los propietarios o inquilinos mediante la aplicación móvil.
        \item Se lleva a cabo la convocatoria.
        \begin{itemize}
            \item Si es una asamblea se procede al siguiente proceso.
            \item Si es una reunión o sesión de directiva se salta al proceso 6.
        \end{itemize}
        \item Los inquilinos o propietarios registran su asistencia en la aplicación móvil.
        \begin{enumerate}
            \item Si no esta registrado en el sistema el presidente o vicepresidente registra su asistencia de su domicilio al que representa.
            \item Si esta registrado en el sistema se procede al siguiente proceso.
        \end{enumerate}
        \item Se verifica la ubicación del propietario o inquilino mediante la geolocalización.
        \begin{itemize}
            \item Si se encuentra en el rango de geolocalización se registra la asistencia y se procede al siguiente proceso.
            \item Si no se encuentra en el rango de geolocalización no se registra la asistencia.
        \end{itemize}
        \item Se tratan los temas de la convocatoria.
        \item Se propone una votación de los temas a tratar.
        \begin{itemize}
            \item Si existe una votación se procede al siguiente proceso.
            \item Si no existe una votación se salta al proceso 14.
        \end{itemize}
        \item Se registra la votación en el sistema.
        \item Se habilita el voto a los propietarios o inquilinos.
        \item El sistema verifica que el usuario que va a votar sea propietario.
        \begin{itemize}
            \item Si es propietario se procede al siguiente proceso.
            \item Si no es propietario no se registra el voto.
        \end{itemize}
        \item Los propietarios registran su voto en la aplicación móvil.
        \item Se verifica la ubicación del propietario mediante la geolocalización.
        \begin{itemize}
            \item Si se encuentra en el rango de geolocalización se registra el voto y se procede al siguiente proceso.
            \item Si no se encuentra en el rango de geolocalización no se registra el voto.
        \end{itemize}
        \item Se muestra el resultado de la votación desde el sistema.
        \item Se finaliza la convocatoria.
        \item Se sube el acta de la convocatoria al sistema.
        \begin{itemize}
            \item Si es una asamblea se procede al siguiente proceso.
            \item Si es una reunión o sesión de directiva se finaliza el proceso.
        \end{itemize}
        \item Se genera el informe de inasistencia de los propietarios o inquilinos desde el sistema.
        \item Se envía el informe de asistencia a los residentes y se registra la respectiva multa en el sistema.
        \item El propietario o inquilino puede presentar una justificación por la inasistencia en las siguientes 24 horas.
        \item Se verifica la justificación presentada.
        \begin{itemize}
            \item Si la justificación es válida se procede a eliminar la multa del sistema y se finaliza el proceso.
            \item Si la justificación no es válida la multa se mantiene y termina el proceso.
        \end{itemize}
    \end{enumerate}


    \item Proceso de administración financiera.
    \begin{enumerate}
        \item Pagos de obligaciones financieras mensuales
        \begin{enumerate}
            \item El inquilino realiza el pago de la mensualidad de sus obligaciones financieras.
            \item El inquilino envía el comprobante de pago a la tesorera.
            \item La tesorera verifica el comprobante de pago.
            \begin{itemize}
                \item Si el comprobante es válido se procede al siguiente proceso.
                \item Si el comprobante no es válido se finaliza el proceso.
            \end{itemize}
            \item La tesorera selecciona la residencia del inquilino o propietario.
            \item El sistema verifica hasta que mes abonó el inquilino en su anterior pago.
            \begin{itemize}
                \item Si el inquilino tiene abonos anteriores el sistema coloca el mes siguiente de manera automática y se procede al siguiente paso.
                \item Si el inquilino no tiene abonos se procede al siguiente paso.
            \end{itemize}
            \item Se registra el pago en el sistema.
            \item Se sube el comprobante de pago al sistema.
            \item Se genera una orden de pago.
            \item Se envía por correo electrónico el respaldo de pago al inquilino o propietario.
        \end{enumerate}

        \item Pagos a proveedores
        \begin{enumerate}
            \item El proveedor envía la factura a la tesorera.
            \item La tesorera verifica la factura.
            \begin{itemize}
                \item Si la factura es válida se procede al siguiente proceso.
                \item Si la factura no es válida se finaliza el proceso.
            \end{itemize}
            \item La tesorera verifíca si el proveedor está registrado en el sistema.
            \begin{itemize}
                \item Si el proveedor está registrado se procede al siguiente proceso.
                \item Si el proveedor no está registrado se le registra en el sistema y se repite el proceso.
            \end{itemize}
            \item La tesorera registra la factura en el sistema.
            \item Se sube la factura al sistema.
        \end{enumerate}

        \item Multas
        \begin{enumerate}
            \item Se registra la multa en el sistema.
            \item Se suben evidencias de la multa al sistema.
            \item Se le notifica al propietario o inquilino la multa mediante la aplicación móvil.
            \item El propietario o inquilino envía el comprobante de pago de la multa.
            \item La tesorera verifica el comprobante de pago.
            \begin{itemize}
                \item Si el comprobante es válido se procede al siguiente proceso.
                \item Si el comprobante no es válido se mantiene la multa y se finaliza el proceso.
            \end{itemize}
            \item Se actualiza el estado de la multa en el sistema.
            \item Se genera una orden de pago.
            \item El sistema envía por correo electrónico el respaldo de pago al propietario o inquilino.
        \end{enumerate}

    \end{enumerate}
\end{itemize}

